\documentclass[twoside,11pt]{TrabajosTex}

\title{Práctica 1}

\author{Autores}

\cita{Defiende tu derecho a pensar, porque incluso pensar de manera errónea es mejor que no pensar.}
\autorcita{Hipátia de Alejandría}

\begin{document}
	\maketitle
	
	\section{Práctica 1}
	
	\subsubsection{Preeliminares}
	Dado que mi ordenador utiliza como sistema operativo Debian 9 (GNU/Linux) la manera de instalar el paquete de \textit{EquationType} ha sido diferente.
	
	\begin{enumerate}
		\item Ejecutamos \textit{cd /usr/local/Wolfram/Mathematica/10.4/AddOns/Packages}
		\item \textit{mkdir EquationType}
		\item \textit{cd EquationType}
		\item \textit{mv ficherodelpaquete .}
	\end{enumerate}

	Luego, cargamos el paquete de \textit{EquationType}:
	
	\begin{lstlisting}
Needs["EquationType`"]
	\end{lstlisting}
	
	Y calculamos, las ecuaciones:
	
	\begin{lstlisting}
EquationType[eqn, {x, y}]
	\end{lstlisting}
	\subsubsection{Ecuación a}
	\begin{empheq}[box={\mybluebox[5pt]}]{equation*}
		u_{xx} - 3 u_{xy} + 2 u _{yy} = 0
	\end{empheq}
	
	Es hiperbólica con discriminante $\frac{1}{4}$
	\subsubsection{Ecuación b}
	\begin{empheq}[box={\mybluebox[5pt]}]{equation*}
		a^u_{xx} + 2au_{xy} + u_{yy} = 0
	\end{empheq}
	
	Es parabólica con discriminante 0.

	\subsubsection{Ecuación c}
	\begin{empheq}[box={\mybluebox[5pt]}]{equation*}
		2u_{xt} + 3u_{tt} = 0
	\end{empheq}
	
	Es hiperbólica con discriminante 1.

	\subsubsection{Eucación d}
	\begin{empheq}[box={\mybluebox[5pt]}]{equation*}
		u_{xx} + x^2 u_{yy} = 0
	\end{empheq}

	Es indeterminada con discriminante $-x^2$.
	\subsubsection{Ecuación e}
	\begin{empheq}[box={\mybluebox[5pt]}]{equation*}
		x^2 u_{xx} + 2xyu_{xy} + y^2 u_{yy} = 0
	\end{empheq}
	
	Es parabólica con discriminante 0.
\end{document}